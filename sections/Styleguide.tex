\section{Style guide und Language Linkage}
\subsection{Style guide}
\begin{itemize}
	\item Programme werden für Programmierer geschrieben, nicht für Compiler
	\item Programmierkonventionen erleichtern einen konsistenten Stil, der auch von anderen verstanden wird
	\item Unwichtiges nicht überbewerten
	\item Alles, was der Compiler entdeckt und mit Warnings meldet (z.B. nicht initialisierte Variablen) muss nicht explizit in den Codierrichtlinien erwähnt werden
	\item Seien Sie konsistent und konsequent!
\end{itemize}

\subsubsection{Namenskonventionen}
\begin{itemize}
	\item Wenige Underscores (\_) verwenden
	\item Keine Namen mit Underscore (\_) beginnen
	\item Zusammengesetzte Namen mit \lstinline{camelCase} betonen, nicht \lstinline{camel_case}
	\item Präfixe bei Variablen sehr zurückhaltend einsetzen oder ganz vermeiden (z.B. \lstinline{i_myVariable} für Integer)
	\item Selbstdefinierte Typen und Klassen mit Grossbuchstaben beginnen
	\item Funktionen und Methoden mit Kleinbuchstaben beginnen
	\item Variablen und Objekte mit Kleinbuchstaben beginnen
	\item Makros (\lstinline{#define}) ausschliesslich mit Grossbuchstaben definieren, Trennung falls notwendig mit Underscores, z.B. \lstinline{MY_UGLY_MACRO}
\end{itemize}

\subsubsection{Typen mit genauer Breite}
\begin{itemize}
	\item Ab C99 werden im Headerfile \lstinline{<stdint.h>} verschiedene Typen mit genauer Breite definiert. Wenn solche Typen benötigt werden, sollen genau diese Namen verwendet werden oder sind allenfalls mittels \lstinline{typedef} zu definieren.
	\item Die Konventionen lauten wie folgt:
	\begin{itemize}
		\item \lstinline{intN_t}, wobei N = 8, 16, 32 für \lstinline{signed int}-Typen \\
		Beispiele: \lstinline{int8_t}, \lstinline{int16_t}, \lstinline{int32_t}
		\item \lstinline{uintN_t}, wobei N = 8, 16, 32 für \lstinline{unsigned int}-Typen \\
		Beispiele: \lstinline{uint8_t}, \lstinline{uint16_t}, \lstinline{uint32_t}
	\end{itemize}
\end{itemize}

\subsection{Language Linkage}
\subsubsection{Plain Old Data Types (POD Types)}
\begin{itemize}
	\item Datentypen, die bereits in C vorhanden sind, werden als POD Types bezeichnet
	\item Dazu gehören \lstinline{char}, \lstinline{short}, \lstinline{int}, \lstinline{long}, jeweils \lstinline{signed} und \lstinline{unsigned}, \lstinline{float} und \lstinline{double}
	\item POD types funktionieren in C++ identisch zu C
\end{itemize}

\subsubsection{Name Mangling}
\begin{multicols}{2}
	Beim Übersetzen in Assembler müssen eindeutige Labels entstehen. In C ist dies kein Problem, da die Funktionsnamen bereits eindeutig sind. Hingegen ist dies in C++ ist nicht der Fall. Deshalb weist der Compiler den Funktionen einen eindeutigen Namen zu (dieser Prozess wird als Mangling bezeichnet).
\columnbreak
  \begin{tabular}{|l|lll|}
  \hline
    C   & {\lstinline!foo()!} & $\rightarrow$ & \_foo \\
  \hline
    C++ & {\lstinline!foo(int)!} & $\rightarrow$ & \_foo\_i \\
        & {\lstinline!foo(double, int)!} & $\rightarrow$ & \_foo\_d\_i \\
        & {\lstinline!MyClass::foo(int)!} & $\rightarrow$ & \_MyClass\_foo\_i \\
  \hline
  \end{tabular}
\end{multicols}

\subsubsection{Einbinden von C-Bibliotheken unter C++}
Wird in C++ eine Funktion \lstinline{foo(int)} aufgerufen welche in einer compilierten C-Library vorliegt, sucht der Compiler nach einer Funktion \_foo\_i. Diese existiert aber nicht da sie \_foo heisst. Deshalb muss dem Compiler mitgeteilt werden, dass es sich um eine C-Funktion handelt.\\

\begin{lstlisting}[style=Cpp]
extern "C" void foo(int);    \\ use C language linkage
extern void goo(int);        \\ use C++ language linkage
void hoo(int);               \\ use C++ language linkage
extern "C++" void koo(int);  \\ use C++ language linkage

//optimale Anwendung von extern "C" (koennen in C und C++ Dateien included werden)
#ifdef __cplusplus
extern "C"
{
#endif
// list multiple prototypes with C linkage, or
// #include C header file(s)
#ifdef __cplusplus
}
#endif
\end{lstlisting}

\subsubsection{C++-Code aus C aufrufen}
\begin{itemize}
	\item C kennt keine Klassen und kennt den this-Pointer nicht
	\item Aus C können keine C++-Elementfunktionen direkt aufgerufen werden
	\item Lösung: C Wrapper Funktionen für C zur Verfügung stellen
	\item Ein Client ruft die Funktion \lstinline{void Philosopher::live()} auf
	\begin{itemize}
		\item Diese Funktion soll im Hintergrund einen eigenen Thread starten
	\end{itemize}
	\item Die Threadfunktion \lstinline{void Philosopher::lifeThread()} soll eine Elementfunktion der Klasse sein
	\begin{itemize}
		\item Dadurch hat diese Funktion alle Vorteile einer Elementfunktion, z.B. Zugriff auf alle Klassenelemente
		\item Sie kann nun in C++ programmiert werden
	\end{itemize}
	\item Gestartet wird der Thread mit der C-Funktion \lstinline{pthread_create()}
\end{itemize}
\begin{lstlisting}{style=cpp}
class Philosopher
{
  public:
    // ...
    void live();        // the philosopher's life
  private:
    // the classes attributes
    pthread_attr_t attr;
    pthread_t tid;     // thread id
    // ...
    void lifeThread();	// the (C++)-thread function
    static void* staticWrapper(void* p) // C Wrapper for pthread_create()
                                        // p must be the this pointer
    {
      static_cast<Philosopher*>(p)->lifeThread();
      return 0;
    }
};
\end{lstlisting}
\begin{lstlisting}{style=cpp}
void Philosopher::live()
{
  pthread_attr_init(&attr);
  pthread_attr_setdetachstate(&attr, PTHREAD_CREATE_JOINABLE);
  pthread_create(&tid, &attr, staticWrapper, this);
}
\end{lstlisting}
