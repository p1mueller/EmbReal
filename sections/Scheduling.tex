%!TEX root = EmbSW1
\section{Scheduling}

\subsection{Tasks in an Embedded System}
The term \glqq task\grqq is a design-time concept.
A real-time system typically has multiple tasks, each of which represents a unit of concurrency.
In general, a task is simply a block of instructions (or actions) to be executed by a processor for a specific purpose.
A task with real-time constraints is called a real-time task.

\subsubsection{Categories}
Most task can be characterized into three categories:
\begin{itemize}
  \item Periodic tasks: A periodic task is a stream of jobs, where the interarrival times between consecutive jobs are almost the same
  \item A sporadic task: A sporadic task is executed in response to events which occur at random instants of time, and the randomness is hard to be characterized by simple probability distribution functions.
  \item Aperiodic tasks: An aperiodic task is a stream of jobs, where the interarrival times between consecutive jobs may follow a known probability distribution function.
\end{itemize}
A system can be a blend of task of different categories. The categories can help to define the appropriate schedule algorithm.

\subsection{Schedulability}
\begin{itemize}
  \item A set of tasks is schedulable if all tasks can meet their \textbf{deadlines} at all times.
  \item \textbf{deadline:}
        \begin{itemize}
          \item latest possible completion time
          \item for periodic tasks this is usually at the same time as the start of the next period
        \end{itemize}
\end{itemize}

\subsection{Multitasking}
\begin{itemize}
  \item Basic problem:
        \begin{itemize}
          \item How are shared resources best allocated?
          \item Different strategies can be used for allocation.
          \item The \textbf{scheduler} makes this allocation.
          \item In the following considerations we assume the allocation of the resource \textbf{CPU}.
        \end{itemize}
\end{itemize}

\subsection{Cooperative Multitasking}
\begin{itemize}
  \item In closed systems, allocation may well be cooperative
  \item Active task decides for itself when to free the processor for other tasks.
  \item An unfair task blocks the others (also a crashed active task)
  \item Select next task via: FCFS, Round Robin, random, priorities, etc.
  \item Very easy to implement
\end{itemize}

\subsubsection{Performance}
\begin{itemize}
  \item Throughput: number of tasks completed per time unit
  \item Utilization: Percentage utilization of a resource.
  \item Average waiting time
  \item more\ldots
\end{itemize}

\subsection{Purpose of Real-Time Scheduling}
\begin{itemize}
  \item All critical time constraints (deadlines, response time) should be met.
  \item In case of emergency the scheduling algorithm has to decide to keep the most critical tasks.
  \item Deadlines of \textbf{less} critical tasks may have to be \textbf{violated}.
\end{itemize}

\subsection{Execution Times}
Scheduling of tasks requires knowledge about the duration of task executions, especially if meeting time constraints has to be guaranteed, as in real-time systems.

\subsubsection[Worst Case Execution Time]{Worst Cate Execution Time (WCET)}
The WCET is the basis for most scheduling algorithms.
The worst case execution time (WCET) is the largest execution time of a program for any input and any initial execution state.

\subsubsection[Best Case Execution Time]{Best Case Execution Time (BCET)}
In contrast there is BCET is the smallest execution time of a program, considering all feasible inputs and initial states.
The BCET is a safe and tight lower bound on the execution time.

\subsection{Scheduler}
A scheduler is a resource broker that is responsible for allocating execution engine resources.
This means it plans the flow/task.
This can be done
\begin{itemize}
  \item statically (fix in programm)
  \item dynamically (changing priority based on rules)
\end{itemize}

\subsection{Approaches to define static Priorities for Tasks}
\subsubsection{Rate-Monotonic Approach: Short Code High priority}
The idea is that short ISR/tasks are executed fast.
Therefore, they block fo a very short time and execution is more guaranteed by priority.
\begin{itemize}
  \item Shortest ISR has highest priority
  \item Shortest flow/task has highest priority
\end{itemize}

\subsubsection{Important Code High Priority}
The idea is important ISR/tasks are executed fast.
\begin{itemize}
  \item Important ISR has highest priority
  \item Important task has highest priority
\end{itemize}

\subsubsection{Blended Approach}
Combination of RMA and ICH
\begin{itemize}
  \item Short task high priority
  \item Important ISR has high priority
\end{itemize}

\subsection{Race condition/Starvation/Deadlock}
\begin{description}
  \item[Race Condition]  The result of an operation depends on the timing behaviour of a specific operation.
  \item[Starvation]      Is a condition where a process never gets to it (it starves).
        The fairness condition states that starvation must be prevented.
  \item[Deadlock]        Is a situation where two processes block each other.
        A deadlock can be avoided by having all processes request the shared resources always in the same order.
\end{description}

\subsubsection{Prevent Race Conditions}
\begin{description}
  \item[Critical Sections] Define sections in which ISR(s) are blocked.
        Pro: Simple, possible on any MCU.
        Cons: Acts blocking.
  \item[Atomic Operations] Command disables ISR for one instruction cycle.
        Atomic commands can not be split in instruction cycles that can be interrupted.
        Atomic by CPU architecture see RISC-V RV32A Atomic Extension.
        Atomic by programming language.
        \textbf{Atomic operation} is a \textbf{foundation} for all other solutions of OS and RTOS.
\end{description}

\subsection{Scheduling Algorithms}

\subsubsection{Sequential One Time Execution}
\columnratio{0.6}
\begin{paracol}{2}
  It is the simplest situation.
  The flow is executed once and is restarted by resetting the system.
  Not really found in real applications.
  \switchcolumn
  \lstinputlisting[style=C]{snippets/Schedule_Algorithms/sequential_one_time_exec.c}
\end{paracol}
\subsubsection{Sequential Continuos Execution}
\begin{paracol}{2}
  The flow is executed infinetely and is restarted by resetting the system.\\
  Typical whem beginning coding. Used for applications which are not really demanding (close to round robin).
  \switchcolumn
  \lstinputlisting[style=C]{snippets/Schedule_Algorithms/sequential_continuos_exec.c}
\end{paracol}

\subsubsection{Round Robin without ISR}
\begin{paracol}{2}
  Cooperative task/flow execution: Every flow runs as long as it has code to execute.
  When the flow is done it returns to the schedule algorithm.
  The schedule algorithm start next flow/task.\\
  This works fine as long as the flows are none blocking and don’t last too long.
  This leads typical to state machines!\\
  \begin{tikzpicture}[auto, node distance=0.1cm, >=latex', font=\tiny,
        block/.style={fill=black!20, draw=black, thick, rectangle, text width=1.0cm, minimum height=0.8cm, align=center}]
    \coordinate (O) at (0, 0);
    \coordinate (S) at ($(O) + (0.1, 0.6)$);

    \draw[->] (O) -- node[pos=1.0, anchor=north east]{time} ++(11, 0);
    \draw[->] (O) -- node[pos=1.0, anchor=north west]{flow/task priority} ++(0, 1.6);
    \draw[densely dotted] (S) ++(-0.1, 0) -- ++(11, 0);

    \node[block, anchor=west](b1) at (S) {\textbf{Flow A}\linebreak Running};
    \node[block, right=of b1](b2) {\textbf{Flow B}\linebreak Running};
    \node[block, right=of b2](b3) {\textbf{Flow C}\linebreak Running};
    \node[block, right=of b3](b4) {\textbf{Flow D}\linebreak Running};
    \node[block, right=of b4](b5) {\textbf{Flow E}\linebreak Running};
    \node[block, right=of b5](b6) {\textbf{Flow A}\linebreak Running};
    \node[block, right=of b6](b7) {\textbf{Flow B}\linebreak Running};
    \node[right=of b7, anchor=north west](b8) {\normalsize \ldots};

\end{tikzpicture}$ $\\
  \switchcolumn
  \lstinputlisting[style=C]{snippets/Schedule_Algorithms/round_robin_no_isr.c}
\end{paracol}

\subsubsection{Round Robin with ISR}
Cooperative task/flow execution: Every flow runs as long as it has code to execute.
When the flow is done it returns to the schedule algorithm.
The schedule algorithm starts next flow/task in queue.
The ISRs give the system more dynamic, less blocking behaviour and smoother state machines.
\columnratio{0.5}
\begin{paracol}{2}
  \lstinputlisting[style=C, lastline=9]{snippets/Schedule_Algorithms/round_robin_isr.c}
  \switchcolumn
  \lstinputlisting[style=C, firstline=11]{snippets/Schedule_Algorithms/round_robin_isr.c}
\end{paracol}
\input{images/Schedule/flow_rr_isr}

\subsubsection{Cooperative task execution round robin with ISR with one time base}
Cooperative task/flow execution: Every flow runs as long as it has code to execute. When the flow is done
it returns to the schedule algorithm. The schedule algorithm start next flow/task.
The ISRs give the system more dynamic, less blocking behaviour and smoother state machines.
This works fine as long as the flows are none blocking and don’t last too long…
Cooperative task execution round robin with ISR: are very common on small microcontrollers
\columnratio{0.5}
\begin{paracol}{2}
  \lstinputlisting[style=C, lastline=6]{snippets/Schedule_Algorithms/round_robin_isr_time.c}
  \switchcolumn
  \lstinputlisting[style=C, firstline=8, lastline=14]{snippets/Schedule_Algorithms/round_robin_isr_time.c}
  \switchcolumn
  \lstinputlisting[style=C, firstline=16, lastline=22]{snippets/Schedule_Algorithms/round_robin_isr_time.c}
  \switchcolumn
  \lstinputlisting[style=C, firstline=24]{snippets/Schedule_Algorithms/round_robin_isr_time.c}
\end{paracol}
\begin{tikzpicture}[auto, node distance=0.1cm, >=latex', font=\tiny,
        block/.style={fill=black!20, draw=black, thick, rectangle, text width=1.0cm, minimum height=0.8cm, align=center},
        timebase/.style={fill=black!50, draw=red, thick, rectangle, rotate=90, text=white, anchor=north west}
    ]
    \coordinate (O) at (0, 0);
    \coordinate (S) at ($(O) + (0.1, 0.6)$);

    \draw[->] (O) -- node[pos=1.0, anchor=north east]{time} ++(17.75, 0);
    \draw[->] (O) -- node[pos=1.0, anchor=north west]{flow/task priority} ++(0, 3.2);
    \foreach \y in {0, ..., 2}{
            \draw[densely dotted] (S) ++(-0.1, \y) -- ++(17.75, 0);
        }
    \draw [decorate,decoration={brace,amplitude=10pt},xshift=4pt,yshift=0pt] (S) ++ (17.75, 2.2) -- ++(0, -1.4) node [black,midway,xshift=15pt,rotate=90,anchor=center] {ISR priorities};

    \node[block, anchor=west](b1) at (S) {\textbf{Flow A}\linebreak Running};
    \node[block, right=of b1](b2) {\textbf{Flow B}\linebreak Running};
    \node[block, above right=1.0cm and 0.1cm of b2.east, anchor=west](b3) {\textbf{ISR A}\linebreak Running};
    \node[block, below right=1.0cm and 0.1cm of b3.east, anchor=west](b4) {\textbf{Flow C}\linebreak Running};
    \node[block, right=of b4](b5) {\textbf{Flow D}\linebreak Running};
    \node[block, right=of b5](b6) {\textbf{Flow E}\linebreak Running};

    \node[timebase] (time1) at (b6.east |- O){WAIT TIME BASE};

    \node[block, right=0.44cm of b6](b7) {\textbf{Flow A}\linebreak Running};
    \node[block, right=of b7](b8) {\textbf{Flow B}\linebreak Running};
    \node[block, right=of b8](b9) {\textbf{Flow C}\linebreak Running};
    \node[block, right=of b9](b10) {\textbf{Flow D}\linebreak Running};
    \node[block, above right=2.0cm and 0.1cm of b10.east, anchor=west](b11) {\textbf{ISR D}\linebreak Running};
    \node[block, below right=2.0cm and 0.1cm of b11.east, anchor=west](b12) {\textbf{Flow E}\linebreak Running};
    \node[timebase] (time2) at (b12.east |- O){WAIT TIME BASE};
    \node[right=0.4cm of b12, anchor=north west](end) {\normalsize \ldots};

    \draw[->, red] (O)++(0, -0.3cm) -- node[black, yshift=-2pt]{timebase} +(b6.east |- O);
\end{tikzpicture}

\subsubsection{Nonpreemptive FIFO Queue with ISR}
Cooperative task/flow execution: Every flow runs as long as it has code to execute. When the flow is
done it returns to the schedule algorithm. The schedule algorithm starts next flow/task in queue.
The ISRs give the system more dynamic, less blocking behaviour and smoother state machines.
\columnratio{0.26, 0.32}
\begin{paracol}{3}
  \lstinputlisting[style=C, lastline=7]{snippets/Schedule_Algorithms/nonpreempt_fifo_queue.c}
  \switchcolumn
  \lstinputlisting[style=C, firstline=9, lastline=15]{snippets/Schedule_Algorithms/nonpreempt_fifo_queue.c}
  \switchcolumn
  \lstinputlisting[style=C, firstline=17]{snippets/Schedule_Algorithms/nonpreempt_fifo_queue.c}
\end{paracol}
\begin{tikzpicture}[auto, node distance=0.1cm, >=latex', font=\tiny,
        block/.style={fill=black!20, draw=black, thick, rectangle, text width=1.0cm, minimum height=0.8cm, align=center}]
    \coordinate (O) at (0, 0);
    \coordinate (S) at ($(O) + (0.1, 0.6)$);

    \draw[->] (O) -- node[pos=1.0, anchor=north east]{time} ++(13, 0);
    \draw[->] (O) -- node[pos=1.0, anchor=north west]{flow/task priority} ++(0, 3.2);
    \foreach \y in {0, ..., 2}{
            \draw[densely dotted] (S) ++(-0.1, \y) -- ++(13, 0);
        }
    \draw [decorate,decoration={brace,amplitude=10pt},xshift=4pt,yshift=0pt] (S) ++ (13.0, 2.2) -- ++(0, -1.4) node [black,midway,xshift=15pt,rotate=90,anchor=center] {ISR priorities};

    \node[block, anchor=west](b1) at (S) {\textbf{Flow A}\linebreak Running};
    \node[block, right=of b1](b2) {\textbf{Flow B}\linebreak Running};
    \node[block, above right=1.0cm and 0.1cm of b2.east, anchor=west](b3) {\textbf{ISR A}\linebreak Running};
    \node[block, below right=1.0cm and 0.1cm of b3.east, anchor=west](b4) {\textbf{Flow C}\linebreak Running};
    \node[block, right=of b4](b5) {\textbf{Flow A}\linebreak Running};
    \node[block, right=of b5](b6) {\textbf{Flow D}\linebreak Running};
    \node[block, right=of b6](b7) {\textbf{Flow E}\linebreak Running};
    \node[block, above right=2.0cm and 0.1cm of b7.east, anchor=west](b8) {\textbf{ISR C}\linebreak Running};
    \node[block, below right=2.0cm and 0.1cm of b8.east, anchor=west](b9) {\textbf{Flow B}\linebreak Running};
    \node[right=of b9, anchor=north west](end) {\normalsize \ldots};
\end{tikzpicture}\\
Flow A start a process.
The ISR injects the execution of flow A into Queue.

\subsubsection{Nonpreemptive Priority Queue with ISR}
Cooperative task/flow execution: Every flow runs as long as it has code to execute.
When the flow is done it returns to the schedule algorithm.
The schedule algorithm starts next highest prioritised flow/task.
The ISRs give the system more dynamic, less blocking behaviour and smoother state machines.
\columnratio{0.245, 0.38}
\begin{paracol}{3}
  \lstinputlisting[style=C, lastline=5]{snippets/Schedule_Algorithms/nonpreempt_prio_queue.c}
  \switchcolumn
  \lstinputlisting[style=C, firstline=7, lastline=13]{snippets/Schedule_Algorithms/nonpreempt_prio_queue.c}
  \switchcolumn
  \lstinputlisting[style=C, firstline=15]{snippets/Schedule_Algorithms/nonpreempt_prio_queue.c}
\end{paracol}
\begin{tikzpicture}[auto, node distance=0.1cm, >=latex', font=\tiny,
        block/.style={fill=black!20, draw=black, thick, rectangle, text width=1.0cm, minimum height=0.8cm, align=center}]
    \coordinate (O) at (0, 0);
    \coordinate (S) at ($(O) + (0.1, 0.6)$);

    \draw[->] (O) -- node[pos=1.0, anchor=north east]{time} ++(13, 0);
    \draw[->] (O) -- node[pos=1.0, anchor=north west]{flow/task priority} ++(0, 5.2);
    \foreach \y in {0, ..., 4}{
            \draw[densely dotted] (S) ++(-0.1, \y) -- ++(13, 0);
        }
    \draw [decorate,decoration={brace,amplitude=10pt},xshift=4pt,yshift=0pt] (S) ++ (13.0, 4.2) -- ++(0, -1.4) node [black,midway,xshift=15pt,rotate=90,anchor=center] {ISR priorities};
    \draw [decorate,decoration={brace,amplitude=10pt},xshift=4pt,yshift=0pt] (S) ++ (13.0, 2.2) -- ++(0, -2.4) node [black,midway,xshift=15pt,rotate=90,anchor=center] {Scheduler priorities};

    \node[block, anchor=west](b1) at ($(S) + (0, 2)$) {\textbf{Flow A}\linebreak Running};
    \node[block, below right=1.0cm and 0.1cm of b1.east, anchor=west](b2) {\textbf{Flow B}\linebreak Running};
    \node[block, above right=2.0cm and 0.1cm of b2.east, anchor=west](b3) {\textbf{ISR A}\linebreak Running};
    \node[block, below right=3.0cm and 0.1cm of b3.east, anchor=west](b4) {\textbf{Flow C}\linebreak Running};
    \node[block, right=of b4](b5) {\textbf{Flow D}\linebreak Running};
    \node[block, above right=2.0cm and 0.1cm of b5.east, anchor=west](b6) {\textbf{Flow A}\linebreak Running};
    \node[block, below right=2.0cm and 0.1cm of b6.east, anchor=west](b7) {\textbf{Flow E}\linebreak Running};
    \node[block, above right=4.0cm and 0.1cm of b7.east, anchor=west](b8) {\textbf{ISR D}\linebreak Running};
    \node[block, below right=3.0cm and 0.1cm of b8.east, anchor=west](b9) {\textbf{Flow B}\linebreak Running};
    \node[below right=1.0cm and 0.1cm of b9.east, anchor=north west](end) {\normalsize \ldots};
\end{tikzpicture}\\
Flow A start a process.
The ISR injects the execution of flow A by setting flag.

\subsubsection{Nonpreemptive Queue: Earliest Deadline First, with ISR}
Cooperative task/flow execution: Every flow runs as long as it has code to execute.
When the flow is done it returns to the schedule algorithm.
The schedule algorithm starts next flow with earliest deadline.
The ISRs give the system more dynamic, less blocking behaviour and smoother state machines.
\columnratio{0.21, 0.4}
\begin{paracol}{3}
  \lstinputlisting[style=C, lastline=5]{snippets/Schedule_Algorithms/early_first.c}
  \switchcolumn
  \lstinputlisting[style=C, firstline=7, lastline=16]{snippets/Schedule_Algorithms/early_first.c}
  \switchcolumn
  \lstinputlisting[style=C, firstline=17]{snippets/Schedule_Algorithms/early_first.c}
\end{paracol}
\input{images/Schedule/early_first}\\
Flow A start a process.
The priority is defined by the deadline of the flows

\subsubsection{Priority based preemptive Task Execution with ISR}
Flows are defined are implemented as tasks an priorities are defined for them.
The scheduler starts tasks if they have the highest priority and they are requested to run.
If a low priority task is running and a higher priority task is requested to run, then it can preempt the lower prioritised task, this gives the system more dynamic.
The ISRs are running as foreground process and give the system more dynamic.\\
\begin{tikzpicture}[auto, node distance=0.1cm, >=latex', font=\tiny,
        block/.style={fill=black!20, draw=black, thick, rectangle, text width=1.1cm, minimum height=0.5cm, align=center}]
    \coordinate (O) at (0, 0);
    \coordinate (S) at ($(O) + (0.1, 0.8)$);

    \draw[->] (O) -- node[pos=1.0, anchor=north east]{time} ++(14.2, 0);
    \draw[->] (O) -- node[pos=1.0, anchor=north west]{flow/task priority} ++(0, 5.2);
    \foreach \y in {0, ..., 4}{
            \draw[densely dotted] (S) ++(-0.1, \y) -- ++(14.2, 0);
        }
    \foreach \n [count=\y] in {A, B, C}{
            \node[block, anchor=east] at ($(S) + (-0.2, 3-\y)$){Flow \n\linebreak Priority};
        }
    \draw [decorate,decoration={brace,amplitude=10pt},xshift=4pt,yshift=0pt] (S) ++ (14.2, 4.2) -- ++(0, -1.4) node [black,midway,xshift=15pt,rotate=90,anchor=center] {ISR priorities};
    \draw [decorate,decoration={brace,amplitude=10pt},xshift=4pt,yshift=0pt] (S) ++ (14.2, 2.2) -- ++(0, -2.4) node [black,midway,xshift=15pt,rotate=90,anchor=center] {Task priorities};

    \node[block, anchor=west](a1) at ($(S) + (0, 2)$) {Running};
    \node[block, right=0cm of a1](a2){Waiting};
    \node[block, right=0cm of a2](a3){Running};
    \node[block, right=0cm of a3](a4){Running};
    \node[block, right=0cm of a4](a5){Not Created};

    \node[block, anchor=west](b1) at ($(S) + (0, 1)$) {Ready};
    \node[block, right=0cm of b1](b2){Running};
    \node[block, right=0cm of b2](b3){Ready};
    \node[block, right=0cm of b3](b4){Ready};
    \node[block, right=0cm of b4](b5){Waiting};

    \node[block, anchor=west](c1) at (S) {Waiting};
    \node[block, right=0cm of c1](c2){Ready};
    \node[block, right=0cm of c2](c3){Ready};
    \node[block, right=0cm of c3](c4){Ready};
    \node[block, right=0cm of c4](c5){Running};

    \node[block, above right=1.0cm and 0.0cm of a5.east, anchor=west](isra){ISR A\linebreak Running};

    \node[block, below right=1.0cm and 0.0cm of isra.east, anchor=west](a6){Not Created};
    \node[block, right=0cm of a6, text width=0.405cm](a7){Not Created};

    \node[block, below=1cm of a6.west, anchor=west](b6){Waiting};
    \node[block, right=0cm of b6, text width=0.405cm](b7){Waiting};

    \node[block, below=1cm of b6.west, anchor=west](c6){Running};
    \node[block, right=0cm of c6, text width=0.405cm](c7){Running};

    \node[block, above right=2.0cm and 0.0cm of a7.east, anchor=west](isrd){ISR D\linebreak Running};

    \node[block, below right=2.0cm and 0.0cm of isrd.east, anchor=west, text width=0.41cm](a8){Not Created};
    \node[block, right=0cm of a8](a9){Not Created};

    \node[block, below=1cm of a8.west, anchor=west, text width=0.41cm](b8){Waiting};
    \node[block, right=0cm of b8](b9){Waiting};

    \node[block, below=1cm of b8.west, anchor=west, text width=0.41cm](c8){Running};
    \node[block, right=0cm of c8](c9){Running};

    \foreach \i in {0, ..., 9}{
            \draw[->] (O) ++ (0.1cm, 0) ++ (\i*1.385cm, 0.2cm) -- ++(1.385cm, 0cm);
        }
    % \node[block, below right=1.0cm and 0.1cm of b1.east, anchor=west](b2) {\textbf{Flow B}\linebreak Running};
    % \node[block, above right=2.0cm and 0.1cm of b2.east, anchor=west](b3) {\textbf{ISR A}\linebreak Running};
    % \node[block, below right=3.0cm and 0.1cm of b3.east, anchor=west](b4) {\textbf{Flow C}\linebreak Running};
    % \node[block, right=of b4](b5) {\textbf{Flow D}\linebreak Running};
    % \node[block, above right=2.0cm and 0.1cm of b5.east, anchor=west](b6) {\textbf{Flow A}\linebreak Running};
    % \node[block, below right=2.0cm and 0.1cm of b6.east, anchor=west](b7) {\textbf{Flow E}\linebreak Running};
    % \node[block, above right=4.0cm and 0.1cm of b7.east, anchor=west](b8) {\textbf{ISR D}\linebreak Running};
    % \node[block, below right=3.0cm and 0.1cm of b8.east, anchor=west](b9) {\textbf{Flow B}\linebreak Running};
    % \node[below right=1.0cm and 0.1cm of b9.east, anchor=north west](end) {\normalsize \ldots};
\end{tikzpicture}\\
Flows are implemented as RTOS tasks with states and priorities.
The schedule decides which tasks is next.
The dispatcher starts the tasks in the system tick intervall.
The RTOS is interrupted by ISRs.

\subsubsection{Priority based preemptive Task Execution and Timeslicing for Tasks same Priority, with ISR}
Flows are defined are implemented as tasks an priorities are defined for them.
The scheduler starts tasks if they have the highest priority and they are requested to run.
If a low priority task is running and a higher priority task is requested to run, then it can pre-empt the lower prioritised task, this gives the system more dynamic.
However Tasks with the same priority are running in cooperative scheduling.
The ISRs are running as foreground process and give the system more dynamic.\\
\begin{tikzpicture}[auto, node distance=0.1cm, >=latex', font=\tiny,
        block/.style={fill=black!20, draw=black, thick, rectangle, text width=1.3cm, inner sep=2pt, minimum height=0.5cm, align=center}]
    \coordinate (O) at (0, 0);
    \coordinate (S) at ($(O) + (0.1, 0.8)$);

    \draw[->] (O) -- node[pos=1.0, anchor=north east]{time} ++(14.9, 0);
    \draw[->] (O) -- node[pos=1.0, anchor=north west]{flow/task priority} ++(0, 5.2);
    \foreach \y in {0, ..., 3}{
            \draw[densely dotted] (S) ++(-0.1, \y) -- ++(14.9, 0);
        }
    \draw [decorate,decoration={brace,amplitude=10pt},xshift=4pt,yshift=0pt] (S) ++ (14.9, 3.2) -- ++(0, -1.4) node [black,midway,xshift=15pt,rotate=90,anchor=center] {ISR priorities};
    \draw [decorate,decoration={brace,amplitude=10pt},xshift=4pt,yshift=0pt] (S) ++ (14.9, 1.2) -- ++(0, -1.4) node [black,midway,xshift=15pt,rotate=90,anchor=center] {Task priorities};
    \node[block, anchor=east, text width=1.0cm] at ($(S) + (-0.2cm, 1cm)$){Flow A\linebreak Priority};
    \node[block, anchor=east, text width=1.0cm] (flowc) at ($(S) + (-0.2cm, 0cm)$){Flow C\linebreak Priority};
    \node[block, left=0.05cm of flowc, anchor=east, text width=1.0cm] {Flow B\linebreak Priority};

    \node[block, anchor=west](a1) at ($(S) + (0, 1)$) {Running};
    \node[block, right=0cm of a1](a2){Waiting};
    \node[block, right=0cm of a2](a3){Running};
    \node[block, right=0cm of a3](a4){Running};
    \node[block, right=0cm of a4](a5){Not Created};

    \node[block, anchor=west](c1) at (S) {B: Waiting\linebreak C: Waiting};
    \node[block, right=0cm of c1](c2){B: Running\linebreak C: Waiting};
    \node[block, right=0cm of c2](c3){B: Ready\linebreak C: Ready};
    \node[block, right=0cm of c3](c4){B: Ready\linebreak C: Ready};
    \node[block, right=0cm of c4](c5){B: Waiting\linebreak C: Running};

    \node[block, above right=1.0cm and 0.0cm of a5.east, anchor=west](isra){ISR A\linebreak Running};

    \node[block, below right=1.0cm and 0.0cm of isra.east, anchor=west](a6){Not Created};
    \node[block, right=0cm of a6, text width=0.565cm](a7){Not Created};

    \node[block, below=1cm of a6.west, anchor=west](c6){B: Waiting\linebreak C: Running};
    \node[block, right=0cm of c6, text width=0.565cm](c7){B:Running\linebreak C:Waiting};

    \node[block, above right=2.0cm and 0.0cm of a7.east, anchor=west](isrd){ISR D\linebreak Running};

    \node[block, below right=2.0cm and 0.0cm of isrd.east, anchor=west, text width=0.565cm](a8){Not Created};
    \node[block, right=0cm of a8](a9){Not Created};

    \node[block, below=1cm of a8.west, anchor=west, text width=0.565cm](c8){\fontsize{4}{0.0} \selectfont B:Running\linebreak C:Waiting};
    \node[block, right=0cm of c8](c9){B: Waiting\linebreak C: Running};

    \foreach \i in {0, ..., 9}{
            \draw[->] (O) ++ (0.1cm, 0) ++ (\i*1.469cm, 0.2cm) -- ++(1.469, 0cm);
        }
    % \node[block, below right=1.0cm and 0.1cm of b1.east, anchor=west](b2) {\textbf{Flow B}\linebreak Running};
    % \node[block, above right=2.0cm and 0.1cm of b2.east, anchor=west](b3) {\textbf{ISR A}\linebreak Running};
    % \node[block, below right=3.0cm and 0.1cm of b3.east, anchor=west](b4) {\textbf{Flow C}\linebreak Running};
    % \node[block, right=of b4](b5) {\textbf{Flow D}\linebreak Running};
    % \node[block, above right=2.0cm and 0.1cm of b5.east, anchor=west](b6) {\textbf{Flow A}\linebreak Running};
    % \node[block, below right=2.0cm and 0.1cm of b6.east, anchor=west](b7) {\textbf{Flow E}\linebreak Running};
    % \node[block, above right=4.0cm and 0.1cm of b7.east, anchor=west](b8) {\textbf{ISR D}\linebreak Running};
    % \node[block, below right=3.0cm and 0.1cm of b8.east, anchor=west](b9) {\textbf{Flow B}\linebreak Running};
    % \node[below right=1.0cm and 0.1cm of b9.east, anchor=north west](end) {\normalsize \ldots};
\end{tikzpicture}\\
Flows are implemented as RTOS tasks with states and priorities.
The schedule decides which tasks is next.
The dispatcher starts the tasks in the system tick intervall.
The RTOS is interrupted by ISRs.

\clearpage
\begin{sidewaystable}
  \begin{tabularx}{\textwidth}{l X X X X X}
    \hline
    Property           & RR no ISR      & RR ISR                   & NE FIFO                  & NE Prio                          & Early First                      \\\hline
    Schedule algorithm & simple         & simple                   & medium                   & medium                           & medium to high                   \\
    Priority of flows  & all the same   & all the same             & all the same, dynamic    & different, static                & different,dynamic                \\
    Priority of ISR    & none           & different, static, high  & different, static, high  & different, high, static          & different, high, static          \\
    Preemption         & none           & ISR preempts flows       & ISR preempts flows       & ISR preempts flows               & ISR preempts flows               \\
    Deterministic      & by flow design & sum of WCET and ISR WCET & sum of WCET and ISR WCET & sum of WCET and ISR WCET         & sum of WCET and ISR WCET         \\
    Occurence          & seldom         & often                    & seldom                   & often, prelimniary stage of RTOS & often, preliminary stage of RTOs \\\hline
  \end{tabularx}
\end{sidewaystable}

\begin{sidewaystable}
  \begin{tabularx}{\textwidth}{l X X}
    \hline
    Property           & Priority Preemptive                                    & Priority Preemptive Timeslicing                        \\\hline
    Schedule algorithm & high                                                   & high                                                   \\
    Priority of flows  & different, fixed or dynamic                            & different, fixed or dynamic                            \\
    Priority of ISR    & different, static, high                                & different, static, high                                \\
    Preemption         & ISR preempts RTOS, RTOS tasks preempt other RTOS tasks & ISR preempts RTOS, RTOS tasks preempt other RTOS tasks \\
    Deterministic      & by flow design                                         & by flow design                                         \\
    Occurence          & often, if knowledge and MCU ressources are available   & often, if knowledge and MCU ressources are available   \\\hline
  \end{tabularx}
\end{sidewaystable}