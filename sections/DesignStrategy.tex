\section[Design Strategy]{Design Strategy on Embedded (Real-time) Software}
\subsection{Targets}
\begin{itemize}
	\item Design should be solution independent for as long as it makes sense (not as long as possible).
	\item Encourage system design, rather than separate designs for mechanics, electronics, firmware, software, etc., which may be conflicting
	\item System specification is ideally done using a unique specification language, not in prose
	\item The specification can be simulated (executed)
	\item Implementations can be easily changed, e.g. from HW to SW or vice versa.
\end{itemize}

\subsection{Requirements for Practical Application}
\begin{itemize}
	\item Methods and tools should not be too technical in system design, i.e. methods should be applicable for electronics, firmware and if possible also mechanics developers
	\item If possible good tool support and automatic synthesis
\end{itemize}

\subsection{Specification Languages}
\begin{itemize}
	\item Formal languages are unique
	\item Examples of specification languages: SystemC, SysML, SpecC, SystemVerilog, Esterel, Matlab/Simulink, Statecharts.
	\item The specification can be compiled and executed
	\item Simulations of the system on a powerful system (e.g. PC) are mostly supported
	\item The executable specification serves as golden reference for future development steps
\end{itemize}

\subsection{Approach}
Approach in a real time embedded design for task and scheduling topics
\begin{enumerate}
	\item Analysis of requirements and split work up into tasks
	\item The system and task model
	\item The scheduling algorithm with related schedule ability test.
	      Implement every flow and measure WCET.
	\item Theoretical and/or empirical performance evaluation of the scheduling algorithm and/or the schedule ability test
\end{enumerate}

\subsubsection{Split Work up into Tasks}
Get the system's timing constraints from \ldots
\begin{itemize}[label=\ldots]
	\item the physics of the system and their implications
	\item the application
\end{itemize}
Data flow charts help to split ub data paths into tasks.