%%%%%%%%%%%%%%%%%%%%%%%%%%%%%%%%%%%%%%%
%% Makros & anderer Low-Level bastel %%
%%%%%%%%%%%%%%%%%%%%%%%%%%%%%%%%%%%%%%%
\makeatletter
%% Makros für Titel, Autor und Datum 
%% Dank diesem Makro stehen Titel, Autor und Datum überall im Dokument zur verfügung
%% Date hat zudem den Default-Wert \today
\def\@Title{}
\def\@Author{}
\def\@Date{\today}
\newcommand{\Title}{\@Title}
\newcommand{\Author}{\@Author}
\newcommand{\Date}{\@Date}
\AtBeginDocument{%
  \let\@Title\@title
  \let\@Author\@author
  \let\@Date\@date
}

%% Makros für den Arraystretch (bei uns meist in Tabellen genutzt, welche Formeln enthalten)
% Default Value
\def\@ArrayStretchDefault{1} % Entspricht der Voreinstellung von Latex

% Setzt einen neuen Wert für den arraystretch
\newcommand{\setArrayStretch}[1]{\renewcommand{\arraystretch}{#1}}

% Setzt den arraystretch zurück auf den default wert
\newcommand{\resetArrayStretch}{\renewcommand{\arraystretch}{\@ArrayStretchDefault}}

% Makro zum setzten des Default arraystretch. Kann nur in der Präambel verwendet werden.
\newcommand{\setDefaultArrayStretch}[1]{%
	\AtBeginDocument{%
		\def\@ArrayStretchDefault{#1}
		\renewcommand{\arraystretch}{#1}
	}
}
\makeatother


%%%%%%%%%%%%%%%%%%%%%%%
%% Wichtige Packages %%
%%%%%%%%%%%%%%%%%%%%%%%
\usepackage[utf8]{inputenc} % UTF-8 unterstützung
\usepackage[english, ngerman]{babel} % Silbentrennung
%\usepackage[automark]{scrpage2} % Header und Footer
\usepackage[automark]{scrlayer-scrpage} %Ersatz für scrpage2 welches veraltet ist
\usepackage{tabularx}

% Für Abbildungen mit mehreren kleinen Bilder
% Doku: http://www.ctan.org/tex-archive/macros/latex/contrib/caption/
\usepackage[justification=centering]{caption}
\usepackage{subcaption}

\ifx \GUARDhsrColors \undefined
\def\GUARDhsrColors{}

\usepackage[table]{xcolor}

\definecolor{HSRWhite}{cmyk}{0,0,0,0}

\definecolor{HSRBlue}{cmyk}{1,0.4,0,0.2}
\definecolor{HSRBlue80}{cmyk}{0.8,0.32,0,0.16}
\definecolor{HSRBlue60}{cmyk}{0.6,0.24,0,0.12}
\definecolor{HSRBlue40}{cmyk}{0.4,0.16,0,0.08}
\definecolor{HSRBlue20}{cmyk}{0.2,0.08,0,0.04}

\definecolor{HSRLightGray}{cmyk}{0,0,0,0.30}
\definecolor{HSRLightGray80}{cmyk}{0,0,0,0.24}
\definecolor{HSRLightGray60}{cmyk}{0,0,0,0.18}
\definecolor{HSRLightGray40}{cmyk}{0,0,0,0.12}
\definecolor{HSRLightGray20}{cmyk}{0,0,0,0.06}

\definecolor{HSRSchwarz}{cmyk}{0,0,0,1}
\definecolor{HSRSchwarz80}{cmyk}{0,0,0,0.8}
\definecolor{HSRSchwarz60}{cmyk}{0,0,0,0.6}
\definecolor{HSRSchwarz40}{cmyk}{0,0,0,0.4}
\definecolor{HSRSchwarz20}{cmyk}{0,0,0,0.2}

\definecolor{HSRHematite}{cmyk}{0.6,1,0.4,0.2}
\definecolor{HSRHematite80}{cmyk}{0.48,0.80,0.32,0.16}
\definecolor{HSRHematite60}{cmyk}{0.36,0.60,0.24,0.12}
\definecolor{HSRHematite40}{cmyk}{0.24,0.40,0.16,0.08}
\definecolor{HSRHematite20}{cmyk}{0.12,0.20,0.08,0.04}

\definecolor{HSRLakeGreen}{cmyk}{0.70,0.30,0.45,0.05}
\definecolor{HSRLakeGreen80}{cmyk}{0.56,0.24,0.36,0.03}
\definecolor{HSRLakeGreen60}{cmyk}{0.42,0.18,0.27,0.02}
\definecolor{HSRLakeGreen40}{cmyk}{0.28,0.06,0.13,0.06}
\definecolor{HSRLakeGreen20}{cmyk}{0.14,0.06,0.09,0.01}

\definecolor{HSRReed}{cmyk}{0.10,0.25,0.45,0.60}
\definecolor{HSRReed80}{cmyk}{0.08,0.20,0.36,0.48}
\definecolor{HSRReed60}{cmyk}{0.06,0.15,0.27,0.36}
\definecolor{HSRReed40}{cmyk}{0.04,0.10,0.18,0.24}
\definecolor{HSRReed20}{cmyk}{0.02,0.05,0.09,0.12}

\definecolor{HSRPetrol}{cmyk}{1,0.18,0,0.45}
\definecolor{HSRPetrol80}{cmyk}{0.64,0.08,0.12,0.32}
\definecolor{HSRPetrol60}{cmyk}{0.48,0.06,0.09,0.24}
\definecolor{HSRPetrol40}{cmyk}{0.32,0.04,0.06,0.16}
\definecolor{HSRPetrol20}{cmyk}{0.16,0.02,0.03,0.08}

\definecolor{HSRBasswood}{cmyk}{0.25,0.05,0.70,0.15}
\definecolor{HSRBasswood80}{cmyk}{0.20,0.04,0.56,0.12}
\definecolor{HSRBasswood60}{cmyk}{0.15,0.03,0.42,0.09}
\definecolor{HSRBasswood40}{cmyk}{0.10,0.02,0.28,0.06}
\definecolor{HSRBasswood20}{cmyk}{0.05,0.01,0.14,0.03}


\fi
\ifx\GUARDmathe\undefined
\def\GUARDmathe{}

\usepackage{amssymb}
% Das mathtools package ist eine Erweiterung zum amsmath package.
% Das amsmath package wird dabei automatisch geladen
\usepackage{mathtools}


% Package mit vielen weiteren Mathe Symbolen
% http://www.ctan.org/tex-archive/fonts/mathabx
\usepackage{mathabx}

% This package defines commands to access bold math symbols. The basic command
% is \bm which may be used to make the math expression in its argument be typeset
% using bold fonts.
\usepackage{bm}

% Package for differentiation operators
\usepackage{esdiff}

\fi
\ifx\GUARDenumitem\undefined
\def\GUARDenumitem{}

\usepackage{enumitem}
\setitemize{noitemsep,topsep=0pt,parsep=0pt,partopsep=0pt}
\setenumerate{noitemsep,topsep=0pt,parsep=0pt,partopsep=0pt}
\fi
\ifx\GUARDlistings\undefined
\def\GUARDlistings{}

%TODO Auf HSR-Farben ändern 
\definecolor{mygreen}{rgb}{0,0.6,0}
\definecolor{mygray}{rgb}{0.5,0.5,0.5}
\definecolor{mymauve}{rgb}{0.58,0,0.82}

\usepackage{listings}
\lstset{ %
    firstnumber=1,
    backgroundcolor=\color{white},   % choose the background color; you must add        \usepackage{color} or \usepackage{xcolor}
    basicstyle=\footnotesize\ttfamily, % the size of the fonts that are used for the code
    breakatwhitespace=false,         % sets if automatic breaks should only happen at whitespace
    breaklines=true,                 % sets automatic line breaking
    captionpos=b,                    % sets the caption-position to bottom
    commentstyle=\color{mygreen},    % comment style
    deletekeywords={...},            % if you want to delete keywords from the given language
    otherkeywords={...},             % if you want to add more keywords to the set
    escapeinside={\%*}{*\%},          % if you want to add LaTeX within your code
    extendedchars=true,              % lets you use non-ASCII characters; for 8-bits encodings only, does not work with UTF-8
    frame=single,	                 % adds a frame around the code
    keepspaces=true,                 % keeps spaces in text, useful for keeping indentation of code (possibly needs columns=flexible)
    keywordstyle=\color{blue},       % keyword style
    language=C++,                    % the language of the code   
    numbers=left,                    % where to put the line-numbers; possible values are (none, left, right)
    numbersep=5pt,                   % how far the line-numbers are from the code
    numberstyle=\tiny\color{mygray}, % the style that is used for the line-numbers
    rulecolor=\color{black},         % if not set, the frame-color may be changed on line-breaks within not-black text (e.g. comments (green here))
    showspaces=false,                % show spaces everywhere adding particular underscores; it overrides 'showstringspaces'
    showstringspaces=false,          % underline spaces within strings only
    showtabs=false,                  % show tabs within strings adding particular underscores
    stepnumber=2,                    % the step between two line-numbers. If it's 1, each line will be numbered
    stringstyle=\color{mymauve},     % string literal style
    tabsize=2,	                     % sets default tabsize to 2 spaces
    %title=\lstname                   % show the filename of files included with         \lstinputlisting; also try caption instead of title
}

\lstdefinestyle{Java}{ numbers=left,
  belowcaptionskip=1\baselineskip,
  breaklines=true,
  frame=L,
  xleftmargin=10pt,
  language=Java,
  showstringspaces=false,
  basicstyle=\footnotesize\ttfamily,
  keywordstyle=\bfseries\color{green!40!black},
  commentstyle=\itshape\color{purple!40!black},
  identifierstyle=\color{blue},
  stringstyle=\color{orange},
  numberstyle=\ttfamily\tiny,
  tabsize=2
}

\lstdefinestyle{SQL}{
  numbers=none,
  belowcaptionskip=1\baselineskip,
  breaklines=true,
  xleftmargin=10pt,
  language=SQL,
  showstringspaces=false,
  basicstyle=\footnotesize\ttfamily,
  keywordstyle=\bfseries\color{green!40!black},
  commentstyle=\itshape\color{purple!40!black},
  identifierstyle=\color{blue},
  stringstyle=\color{orange},,
  tabsize=2
}

\lstdefinestyle{C}{
  numbers=left,
  belowcaptionskip=1\baselineskip,
  breaklines=true,
  frame=L,
  xleftmargin=10pt,
  language=C,
  showstringspaces=false,
  basicstyle=\footnotesize\ttfamily,
  keywordstyle=\bfseries\color{green!40!black},
  commentstyle=\itshape\color{purple!40!black},
  identifierstyle=\color{blue},
  stringstyle=\color{orange},
  numberstyle=\ttfamily\tiny,
  tabsize=2
}

\lstdefinestyle{Cpp}{
  numbers=left,
  belowcaptionskip=1\baselineskip,
  breaklines=true,
  frame=L,
  xleftmargin=10pt,
  language=C++,
  showstringspaces=false,
  basicstyle=\footnotesize\ttfamily,
  keywordstyle=\bfseries\color{green!40!black},
  commentstyle=\itshape\color{purple!40!black},
  identifierstyle=\color{blue},
  stringstyle=\color{orange},
  numberstyle=\ttfamily\tiny,
  tabsize=2
}

\lstdefinestyle{Cppunit}{
    belowcaptionskip=1\baselineskip,
    %frame=L,
    xleftmargin=\parindent,
    language=C++,
    keywordstyle=\bfseries\color{blue},
    keywordstyle=[2]\bf\color{black}, %not sure why \bf works, but it does
    commentstyle=\itshape\color{mygreen},
    identifierstyle=\color{black},
    stringstyle=\color{gray},
    keywords=[2]{  %Cpp Unit Keywords
        CPPUNIT_ASSERT,
        CPPUNIT_TEST,
        CPPUNIT_TEST_EXCEPTION,
        CPPUNIT_TEST_END,
        CPPUNIT_TEST_SUITE,
        CPPUNIT_TEST_SUITE_REGISTRATION,
        CPPUNIT_TEST_SUITE_END},
}

\lstdefinestyle{CppQT}{
    belowcaptionskip=1\baselineskip,
    %frame=L,
    xleftmargin=\parindent,
    language=C++,
    keywordstyle=\bfseries\color{blue},
    keywordstyle=[2]\bfseries\color{red},
    commentstyle=\itshape\color{mygreen},
    identifierstyle=\color{black},
    stringstyle=\color{gray},
    keywords=[2]{           % qt-Keywords
        Qt,
        SIGNAL,
        SLOT,
        QApplication,
        QDialog,
        QGridLayout,
        QPushButton,
        QLabel,
        QVBoxLayout,
        QHBoxLayout,
        QWidget,
        QGroupBox,
        QFont,
        QLineEdit,
        QRadioButton,
        QPen,
        QRect,
        QPaintEvent,
        QBrush,
        QPixmap,
        QPainter,
        QString,
        QPoint,
        update()},
}

\lstdefinestyle{Cdoxy}{
    belowcaptionskip=1\baselineskip,
    %frame=L,
    xleftmargin=\parindent,
    language=C++,  
    keywordstyle=\bfseries\color{blue},
    commentstyle=\itshape\color{mygreen},
    identifierstyle=\color{black},
    stringstyle=\color{gray},
    otherkeywords={           % DoxygenKeywords
        ...,
        ....,
        @mainpage,
        @file,
        @author,
        @version,
        @date,
        @bug,
        @brief,
        @extended,
        @param,
        @return,
        @warning,
        @note,
        @see},
}

\lstdefinestyle{Csharp}{
  numbers=left,
  belowcaptionskip=1\baselineskip,
  breaklines=true,
  frame=L,
  xleftmargin=10pt,
  language=[Sharp]C,
  showstringspaces=false,
  basicstyle=\footnotesize\ttfamily,
  keywordstyle=\bfseries\color{green!40!black},
  commentstyle=\itshape\color{purple!40!black},
  identifierstyle=\color{blue},
  stringstyle=\color{orange},
  numberstyle=\ttfamily\tiny,
  tabsize=2
}

\lstdefinestyle{Matlab}{
  numbers=left,
  belowcaptionskip=1\baselineskip,
  breaklines=true,
  frame=L,
  xleftmargin=10pt,
  language=Matlab,
  showstringspaces=false,
  basicstyle=\footnotesize\ttfamily,
  keywordstyle=\bfseries\color{blue},
  commentstyle=\itshape\color{mygreen},
  identifierstyle=\color{black},
  stringstyle=\color{orange},
  numberstyle=\ttfamily\tiny,
  tabsize=2,
  otherkeywords={
      solve,
      int,
      double,
      syms,
      interp1}
}

\lstdefinestyle{VHDL}{
  numbers=left,
  belowcaptionskip=1\baselineskip,
  breaklines=true,
  frame=L,
  xleftmargin=0pt,
  language=VHDL,
  showstringspaces=false,
  basicstyle=\footnotesize\ttfamily,
  keywordstyle=\bfseries\color{green!40!black},
  commentstyle=\itshape\color{purple!40!black},
  identifierstyle=\color{blue},
  stringstyle=\color{orange},
  numberstyle=\ttfamily\tiny,
  tabsize=2
}

\lstdefinestyle{ASM}{
  numbers=left,
  belowcaptionskip=1\baselineskip,
  breaklines=true,
  frame=L,
  xleftmargin=10pt,
  language=[x86masm]Assembler,
  showstringspaces=false,
  basicstyle=\footnotesize\ttfamily,
  keywordstyle=\bfseries\color{green!40!black},
  commentstyle=\itshape\color{purple!40!black},
  identifierstyle=\color{blue},
  stringstyle=\color{orange},
  numberstyle=\ttfamily\tiny,
  tabsize=2
}

\lstdefinestyle{R}{
  numbers=left,
  belowcaptionskip=1\baselineskip,
  breaklines=true,
  frame=L,
  xleftmargin=10pt,
  aboveskip=-5pt,
  language=R,
  showstringspaces=false,
  basicstyle=\footnotesize\ttfamily,
  keywordstyle=\bfseries\color{green!40!black},
  commentstyle=\itshape\color{purple!40!black},
  identifierstyle=\color{blue},
  stringstyle=\color{orange},
  numberstyle=\ttfamily\tiny,
  tabsize=1
}
\fi


% Seitenränder für Formelsammlungen 
\usepackage[left=1cm,right=1cm,top=0.5cm,bottom=0.5cm,includeheadfoot]{geometry}

\usepackage[plainpages=false,pdfpagelabels]{hyperref}
\hypersetup{
	pdfstartview={FitH}, % fits the width of the page to the window
	pdfauthor={\Author},
	pdfcreator={\Author},
	pdfproducer={\Author},
	pdftitle={\Title},
	colorlinks=true,
	linkcolor=HSRBlue,
	citecolor=HSRReed,
	filecolor=HSRLake,
	urlcolor=HSRHematite
}

\usepackage{longtable}
\usepackage[europeanresistors,americaninductors]{circuitikz}
\usepackage{multirow} % Create tabular cells spanning multiple rows
\usepackage{multicol} % In­ter­mix sin­gle and mul­ti­ple columns
\usepackage{rotating} % Rotation tools, including rotated fullpage floats


%%%%%%%%%%%%%%%%%%%%%%%%%%%%%%%%%%%
%% Layout der Kopf und Fusszeile %%
%%%%%%%%%%%%%%%%%%%%%%%%%%%%%%%%%%%
\deftriplepagestyle{zusammenfassung}[0pt][0.5pt]
{\Title}	% Kopfzeile innen
{\headmark}	% Kopfzeile mitte
{\pagemark}	% Kopfzeile aussen
{\Author}	% Fusszeile innen
{}			% Fusszeile mitte
{\Date}	% Fusszeile aussen
\pagestyle{zusammenfassung}

% Nummerierte und unnummerierte paragraph headings
\setcounter{secnumdepth}{5}

\RedeclareSectionCommands[
	afterskip=1em
]{paragraph,subparagraph}

\renewcommand*{\paragraphformat}{\theparagraph\autodot\enskip}
\renewcommand*{\subparagraphformat}{\thesubparagraph\autodot\enskip}


%%Makros%%%%%%%%%%%%%%%%%%%%%%%%%%%%%%%%%%%%%%%%%%%  
% Command for images in table
\newcommand\tabImg[2][]{%
	\raisebox{0pt}[\dimexpr\totalheight+\dp\strutbox\relax][\dp\strutbox]{%
		\includegraphics[#1]{#2}%
	}%
}

% Makros für Verweise auf ein Buch oder Skript
\newcommand{\buch}[1]{\texorpdfstring{$_{\textcolor{HSRLakeGreen}{\mbox{\small{#1}}}}$}{}}
\newcommand{\buchSeite}[1]{\texorpdfstring{\ensuremath{_{\textcolor{red}{\mbox{\small{ S#1}}}}}}{}}
\newcommand{\skript}[1]{\texorpdfstring{$_{\textcolor{HSRReed}{\mbox{\small{#1}}}}$}{}}
\newcommand{\formelbuch}[1]{$_{\textcolor{red}{\mbox{\small{S#1}}}}$}

% Zeilenhöhe Tabellen:
\newcommand{\arraystretchOriginal}{1.5}
\renewcommand{\arraystretch}{\arraystretchOriginal}

\setlength{\parindent}{0pt}

% Todo command
\newcommand{\todo}[1]{\textbf{\color{red}{TO DO: #1}}}